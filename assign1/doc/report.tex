\documentclass{report}
\usepackage[utf8]{inputenc}
\usepackage{algorithm}
\usepackage{graphicx}
\usepackage{amsmath}
\usepackage[noend]{algpseudocode}

\title{\includegraphics[scale=0.3]{logo.png} \\ \textbf{A Performance Evaluation of Programming Languages Operating in Single Core Instructions}}
\author{Parallel and Distributed Computing \\ Bachelors in Informatics and Computer Engineering \\ \\ 3L.EIC01\_G5   \\ Joel Fernandes up201904977@up.pt \\ Mário Travassos up201905871@up.pt \\ Tiago Rodrigues up201907021@up.pt }

\graphicspath{{./images/}}

\begin{document}
    \maketitle

    \section*{Introduction}

    \paragraph{}This project intends to show and evaluate the effect of processor performance when accessing large amounts of data, performing the same instructions multiple times. In this study, the product of two matrices was used as the base calculation.

    \paragraph{}Also, a comparison of how different programming languages interact with memory and impact the processor speed is shown. It is important to highlight that these tests were performed on a single core, so no parallelism optimizations are made.

    \paragraph{}Finally, performance measures were made using the Performance API (PAPI), which will be analyzed and discussed in further detail.

    \section*{Problem Description}

    \paragraph{}The problem used to evaluate the performance was the matrix multiplication. It was chosen because the amount of instructions does not impact performance tremendously, with the greatest bottleneck being memory access.

    \paragraph{}That way, we can measure more truthfully how much time does the processor spend accessing memory, and the impact that cache hits and misses have on a program.

    \paragraph{}Even though the main intention is to measure memory access performance, we also could see how some improvements in the algorithms used could make the processing time differ.

    \paragraph{}The first improvement was to multiply by line instead of the usual matrix multiplication. Then, a further improvement made was multiplying by block, which is shown to reduce running times.

    \newpage

    \section*{Algorithm Analysis}

    \subsection*{Normal Multiplication}

    \paragraph{}The first algorithm developed calculates the matrix product the way that students are traditionally taught in their Algebra class. In this algorithm, values are multiplied sequentially, with the first element being the dot product between the first row from the first matrix and the first column from the second matrix, and so on.

    \paragraph{}The following pseudocode details how the algorithm works:

    \begin{algorithm}
      \caption{Regular Multiplication}\label{euclid}
      \begin{algorithmic}[1]
        \Procedure{Regular Multiplication}{}
        \While{$i < firstMatrixSize$}
        \While{$j < secondMatrixSize$}
        \State $temp \leftarrow 0$
        \While{$k < firstMatrixSize$}
        \State $temp \leftarrow temp + firstMatrix_{i, k} + secondMatrix_{k, j}$
        \State $k$++
        \EndWhile
        \State $thirdMatrix_{i, j} \leftarrow temp$
        \State $j$++
        \EndWhile
        \State $i$++
        \EndWhile
        \EndProcedure
      \end{algorithmic}
    \end{algorithm}

    \paragraph{}Although it is simple to implement and easy to understand as it follows the mathematical definitions, it is not very performant as matrix sizes increase drastically.

    \subsection*{Line Multiplication}

    \paragraph{}In this version of the algorithm, all that is done is a change in the order of operations. Instead of performing the calculations based on the result matrix, we perform them based on the rows of the second matrix. This means that first, all values of the first row of the second matrix are ``used'', and only then does it change to the second one, and so on.

    \paragraph{}To further clarify this, the following pseudocode shows how it works:

    \begin{algorithm}
      \caption{Line Multiplication}\label{euclid}
      \begin{algorithmic}[1]
        \Procedure{Line Multiplication}{}
        \While{$i < firstMatrixSize$}
        \While{$j < secondMatrixSize$}
        \While{$k < firstMatrixSize$}
        \State $thirdMatrix_{i, k} \leftarrow thirdMatrix_{i, k} + firstMatrix_{i, j} + secondMatrix_{j, k}$
        \State $k$++
        \EndWhile
        \State $j$++
        \EndWhile
        \State $i$++
        \EndWhile
        \EndProcedure
      \end{algorithmic}
    \end{algorithm}

    \subsection*{Block Multiplication}

    \section*{Performance Evaluation}

    \subsection*{Metrics Used}

    \paragraph{}To evaluate the performance of each algorithm, 5 metrics were used to determine both speed and memory performance. To measure speed, the running time of the calculations was measured, in milliseconds. To measure memory access, the Performance API (PAPI) was used, to measure cache hits and misses, both to the L1 and the L2 cache.

    \subsection*{Results Analysis}

    \paragraph{}Cringe

    \section*{Conclusion}

    \paragraph{}The way and order in which we perform computations can have a great impact on performance. The use of the proper algorithms, even in single core programs, can have a significant impact on processing time as well. Even more, the programming language used and the way it handles memory can also affect the time, as we have shown in this report.

    \paragraph{}The result turns what to be what was expected, with a language designed for speed like C++ beating out Java in every metric. Mention cache hits and misses whenever they are available

    \paragraph{}With this in mind, the report is satisfactory, as it concludes what was thought to be true beforehand, and it contributed to our knowledge of how machine instructions, order of operations and memory access can have a significant impact on the speed of programs.

\end{document}
